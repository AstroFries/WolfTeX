\documentclass[a4paper, 12pt, titlepage = false]{article}
\usepackage{ctex} 
\usepackage{tikz}
\usepackage{amsfonts}
\usepackage{amsmath}
\title{HW12}
\author{Astrofries}
\date{\today}
\begin{document}
    \maketitle
    \section{}
    \[\mathbb E_{p_1,p_2}(X) = 1\cdot p_1 + 2\cdot p_2 + 3(1-p_1-p_2) = 3 - 2p_1 - p_2 \]
    \[\mathbb E_{p_1,p_2}(X^2) = 1^2\cdot p_1 + 2^2\cdot p_2 + 3^2 (1-p_1-p_2) = 9 - 8p_1 - 5p_2\]
    根据矩估计法,将样本的$\mathbb E$认为是的$\mathbb E_{p_1,p_2}$联立得知:
    \[\begin{pmatrix}
        \hat p_1\\
        \hat p_2
    \end{pmatrix} = 
    \frac{1}{2} 
    \begin{pmatrix}
        -5&1\\
        8 &-2
    \end{pmatrix}
    \begin{pmatrix}
        \mathbb E(X)\\
        \mathbb E(X^2)
    \end{pmatrix} + 
    \begin{pmatrix}
        3\\
        -3
    \end{pmatrix}\]
    %
    \section{}
    令$\theta_1 = \mu,\theta_2 = \sigma^2$
    \[\mathbb E_{\theta_1,\theta_2}(a\overline{X_1} + (1-a)\overline{X_2}) = a\mathbb E_{\theta_1,\theta_2}(\overline{X_1})+(1-a)\mathbb E_{\theta_1,\theta_2}(\overline{X_2}) = \theta_1\]
    因此其是无偏估计;
    另外:
    \begin{align*}
        \mathbb D_{\theta_1,\theta_2}(a\overline{X_1}+(1-a)\overline{X_2}) &= \mathbb E_{\theta_1,\theta_2}(a\overline{X_1-\mu}+(1-a)\overline{X_2-\mu})^2\\
        &= \mathbb E_{\theta_1,\theta_2}(a\overline{X_1-\mu})^2+\mathbb E_{\theta_1,\theta_2}(a(1-a)\overline{X_1-\mu}\ \overline{X_2-\mu})\\
        &\quad +\mathbb E_{\theta_1,\theta_2}((1-a)\overline{X_2-\mu})^2\\
        &= a^2 \frac{\sigma^2}{n_1} +(1-a)^2\frac{\sigma^2}{n_2} \\
        &= \sigma^2\left[\left( \frac{1}{n_1} +\frac{1}{n_2} \right)a^2 -\frac{2}{n_2}a+\frac{1}{n_2}   \right]
    \end{align*}
    在$a = \dfrac{n_1}{n_1+n_2}$时取到最小。 
    \section{}
    \subsection{}
    \[\mathbb E_{\theta}(X) = \int_0^1 xf(x)\mathrm dx = \int_0^\theta \frac{x}{2\theta}\mathrm dx + \int_{\theta}^1\frac{x}{2(1-\theta)} \mathrm dx = \frac{1}{2} \theta + \frac{1}{4} \]
    于是
    \[\hat \theta = 2\mathbb E(X) - \frac{1}{2} \]
    %$\hat \theta = 2\mathbb E(X) - \frac{1}{2} $
    %Residue[1/z,{z,0}]
    %1+1
    % DSolve[x[t] == x'[t],x[t],t]
    %D[x^3 * Sin[x], x]
    %Integrate[Exp[-k*x^2], {x, -Infinity, Infinity}]
    
    \subsection{}
    \[\mathbb E_{\theta}(X_k^2) = \int_{0}^{\theta}\frac{x^2}{2\theta}\mathrm dx +\int_{\theta}^{1}\frac{x^2}{2(1-\theta)}\mathrm dx =\frac{1}{6} (2\theta^2+\theta+1)\]
    \begin{align*}
        \mathbb E_{\theta}(\hat{\theta}(X)) &= \mathbb E_{\theta}(4\overline X^2) \\
        &= \frac{4}{n^2} \mathbb E_{\theta}\left[\left(\sum_{k=1}^{n}X_k\right)^2\right]\\
        &= \frac{4}{n^2} \mathbb E_{\theta}\left[\sum_{k=1}^{n}X_k^2 + \sum_{1\le k<m\le n}X_mX_n\right]\\
        &= \frac{4}{n^2} \left[n\cdot \frac{1}{6} (2\theta^2 + \theta + 1) + \frac{n(n-1)}{2}\left(\frac{1}{2} \theta + \frac{1}{4} \right)^2 \right]\neq \theta^2
    \end{align*}
    因此是有偏估计。
    \section{}
    \subsection{}
    矩估计:
    \[\overline X = \frac{1}{2}\theta\Rightarrow \hat \theta = 2\overline X \]\par
    最大似然估计:\\
    只需要让$[\theta,\theta +|\theta| = 0]$区间长度最小且包裹所有样本点即可,因此$\hat{\theta} = \min_{1\le k \le n}{X_k}$
    \subsection{}
    矩估计:
    \[\overline X = \frac{3}{2} \theta \Rightarrow \hat{\theta} = \frac{2}{3} \overline X\]\par
    最大似然估计:
    还是让$[\theta,\theta +|\theta| = 2\theta]$区间最小且包裹所有样本点,于是$\hat\theta = \frac{1}{2} \max X$
    \section{}
    \subsection{}
    矩估计:
    \[\mathbb E_{\theta}(X) = \theta + \sigma\Rightarrow \hat{\theta} = \overline X - \sigma\]\par
    最大似然估计:\\
    只考虑$\theta>\min X$的情况(反之似然函数=0)
    \[L(\theta) = \prod_{k=1}^{n}\frac{1}{\sigma}\exp(-(X_k-\theta)/\sigma)= \exp(n\theta/\sigma)\prod_{k=1}^{n}\frac{1}{\sigma}\exp(-X_k/\sigma)\]
    于是应该取$\hat \theta = \min X$
    \subsection{}
    矩估计:
    \[\mathbb E_{\theta} (\overline X - \sigma) = \sum_{k=1}^{n}\mathbb E_{\theta}(X_k) - \sigma = \theta \]
    所以是无偏的。\par
    最大似然估计:\\
    假设
    \[f_{\min_{1\le i\le k}X_i} (t) = \frac{k}{\sigma}\exp(-k(t-\theta)/\sigma) \]
    $k=1$成立。$k>1$:
    \begin{align*}
        f_{\min_{1\le i\le k}X_i} (t) &= k f_{X_k}(t)\int_t^{+\infty}f_{\min_{1\le i\le k-1}X_i} (s)\mathrm ds \\
        &= \frac{k}{\sigma}\exp(-(t-\theta)/\sigma)\cdot \exp(-(k-1)(t-\theta)/\sigma)\\
        &= \frac{k}{\sigma}\exp(-k(t-\theta)/\sigma)\\
    \end{align*}
    第一行等式右边的k意为$1\sim k$中任意一个都可能最小,但他们是等概率的,所以这里只计算$X_k$最小的概率密度分布,然后乘k得到总概率密度分布。
    由数学归纳法知假设对任意k成立,因此
    \[\mathbb E_{\theta}(\min_{1\le k \le n}X_k) = \theta + \frac{\sigma}{n} \neq \theta\]
    所以最大似然估计是有偏的,应该修正为$\hat \theta = \min_{1\le k \le n}X_k - \frac{\sigma}{n}$ 才为无偏。
    \section{}
    \subsection{}
    
    显然所有x的概率都关于$\mu$单调增,于是似然函数也是单增的(在$mu\le \min X$时),于是最大似然估计给出
    \[\hat \mu = \min_{1\le k \le n}X_k\]
    仿照5知道其是有偏的,应该修正为$\hat \mu ^{**} = \min_{1\le k \le n}X_k - \frac{1}{n}$才无偏。
    \subsection{} 
    推导矩估计和证明其为无偏估计的过程与5中相同。\par
    由于$\hat\mu^{**}-\mu$服从$\mathrm{Exp}(n)$,
    \[\mathrm{Var} \hat \mu^{**} = \frac{1}{n^2} ;\mathrm{Var} \hat \mu = \frac{1}{n} \mathrm{Var} X_1 = \frac{1}{n} \]
    因此$\hat \mu^{**}$更有效
\end{document}

%49f98e11-73f8-41ae-8d18-7833ad2e329f